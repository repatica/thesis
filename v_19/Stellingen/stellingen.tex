\documentclass[a4paper,10pt]{article}
\usepackage[utf8]{inputenc}

%opening
%\title{Stellingen behorend bij het proefschrift}
%\date{}


\begin{document}
%\maketitle

\thispagestyle{empty}

\begin{center}
Stellingen behorend bij het proefschrift
\end{center}
\begin{center}
TREES, AGREEMENT FORESTS AND TREEWIDTH: COMBINATORIAL
ALGORITHMS FOR CONSTRUCTING PHYLOGENETIC NETWORKS
\end{center}


\begin{enumerate}
  \item The study of phylogenetic networks can contribute to a better understanding of
long-standing open problems in complexity. \emph{(This thesis, Chapter 2)}

\item The Cycle Killer software shows that some NP-hard problems can be approximated very well
in practice by solving a sequence of ``easier'' NP-hard problems exactly. \emph{(This thesis, Chapters 3 and 4)}

  \item The Minimum hybridization problem with three input trees behaves completely differently to the Minimum Hybridization problem with two input trees. \emph{(This thesis, Chapter 5)}
  
 \item If we are going to use treewidth as a proxy measure for phylogenetic dissimilarity, then we'll need more complex auxiliary graph structures than display graphs. \emph{(This thesis, Chapter 6)}
  
 \item Common chains rarely occur in nature, and long common chains simply never occur,
so mathematicians should stop worrying about them.

  \item Bioinformatics offers a wealth of beautiful and mathematically challenging combinatorial optimization problems. They deserve more appreciation than they have received so far from the combinatorial optimization community. 
  
  \item It is easy to say that biologists should never try to model reticulate evolution (caused by phenomena such as such as hybridization, recombination and horizontal gene transfer) with a tree. But this assumes it is easy to distinguish between reticulate evolution and ``noisy'' treelike evolution...and this assumption is wrong.
    
  \item The study of phylogenetic networks can generate new insights into the ancestral relations of pathogens and help with identifying the genes that cause diseases in humans and other animals. \emph{(Valorization)}
  
  \item ``In the realm of ideas everything depends on enthusiasm; in the real world all rests on perseverance.'' (Johann Wolfgang von Goethe, 1749-1832) 
  
  \item ``Children need time to think things over. You must give them this time.'' (Martine Delfos, NRC Handelsblad, 15 Augustus 2015, translation by the candidate.) Remember to give it to the adults too.
  
\end{enumerate}

\begin{center}
Nela Leki\'c, December 2015.
\end{center}

\end{document}
