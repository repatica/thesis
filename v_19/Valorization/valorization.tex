\documentclass[a4paper,10pt]{article}
\usepackage[utf8]{inputenc}

\setlength{\parindent}{4em}
\setlength{\parskip}{1em}
\renewcommand{\baselinestretch}{2.0}

\title{Valorization}
\date{}

\begin{document}
\maketitle



Valorization is the process of creating value from knowledge since the knowledge, as we all know, does not have a value in itself. It is important to argue existence of commercial benefits to be able to make scientific theoretic work valuable. In this way we save four years of work from being useless and incomplete. 

Reconstructing evolutionary relationships between species is something that concerns everyone, not ``just'' academics. The contribution of this research to the society is in developing methods that help locate the origin of pathogens. This is important because understanding the genetic origin of pathogens can help stop the spread of outbreaks and possibly prevent similar ones happening in the future. For example, phylogenetic analyses have helped to identify the origin of fungal outbreaks
on Vancouver Island and the Pacific Northwest, caused by Cryptococcus gattii, which
infected hundreds of animals and humans. Phylogeny was also used to help locate the outbreak of the SARS virus. 

Furthermore, recombination is a relevant factor in recognizing diseases caused by genes. To understand the inheritance of diseases and their spread to the descendants, we need to understand how chromosomes crossover. The search for genes that cause genetic diseases or other important traits is called association mapping. Accurately reconstructing evolutionary histories is essential for this task.

In addition to biology, examples of valorization can be seen in other application areas such as linguistics, which also studies evolutionary scenarios (of languages).















\end{document}
